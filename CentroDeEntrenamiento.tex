\documentclass{beamer}


\usetheme{Warsaw}

%\usepackage{babel}
\usepackage{pgf}
\usepackage{tikz}
\usepackage{pgflibraryshapes}
\usepackage{transparent}
\usepackage{url}
\usepackage{hyperref}
%@preamble{ "\@ifundefined{url}{\def\url#1{\texttt{#1}}}{}" }
%\usepackage[utf8x]{inputenc}
%\usepackage{default}

\titlegraphic{\includegraphics[scale=.195]{icpc.png}}

\title[Concursos de Conocimientos:\textit{Experiencia tunera}]{{\Huge Concursos de Conocimientos}\\\textit{\LARGE Experiencia tunera}}
\author[IV TCAC Frank Arteaga Salgado]{\Large IPVCE Luis Urquiza Jorge\\{\Large Frank Arteaga Salgado}\\\texttt{\small farteaga@ipvce.lt.rimed.cu}}
\date{\small 8 de Noviembre de 2014}


\begin{document}

\begin{frame}
{\Large V Taller Caribe\~no de Aprendizaje Colaborativo }
 \titlepage
\end{frame}


\begin{frame}{\LARGE Olimpiadas Internacionales de Conocimientos}
\LARGE
  \begin{itemize}
    \item<1-> Anualmente participan la gran mayor\'ia de los pa\'ises.
	
    \item<2->  La UNESCO, \textbf{empresas} y universidades las auspician para propagar
el inter\'es en ciencias b\'asicas.

     \item<3-> Las ciencias est\'an en todos los \'ambitos
de la vida: salud, comunicaciones, transporte, econom\'ia, etc.
  \end{itemize}
\end{frame}


\begin{frame}{\LARGE C\'omo se inserta Cuba}
\Large
\begin{itemize}
 \LARGE
  \item Cuba participa desde 1973 cuando fue el primer pa\'is latinoamericano
  en participar en la IMO.
  \pause
  \item Fue sede: IMO 1987, IPhO 1991, OIBF 2003, OIBQ 2009.
  \pause
  \item Realiza las Olimpiadas Populares y los Concursos Nacionales para conformar la
  Preselecci\'on Nacional.
\end{itemize}  
\end{frame}

\begin{frame}{\LARGE Las Tunas}
\Large
\begin{block}{Resultados de los cuatro \'ultimos Concursos Nacionales}
\pause
\begin{center}
\begin{tabular}{|l|l|l|}
\hline
 Curso & Puntos & Lugar\\\hline
2010-2011 & 71 (124 $1^{er}$ lugar) & Sexto\\\hline
2011-2012  & 113 (128 $1^{er}$ lugar) & Tercero\\\hline
2012-2013  & 140 & \textbf{Primero}\\\hline
2013-2014 & 139 & \textbf{Primero} \\\hline
\end{tabular}  
\end{center}
\end{block}  

%Que hizo posible esto?
\pause    
\begin{block}{\ }
	\LARGE Creaci\'on de un Centro de Entrenamiento Provincial en el IPVCE Luis Urquiza Jorge.
\end{block}
\end{frame}

\begin{frame}{\LARGE Las Tunas: Centro de Entrenamiento}
  \LARGE
  \begin{block}{Misi\'on del Centro de Entrenamiento} 
	Concentrar a los estudiantes talentosos que ingresen al mismo
	para \textbf{prepararlos intensiva y sistem\'aticamente} en las asignaturas afines.
  \end{block}
  
\end{frame}

\begin{frame}{\LARGE Las Tunas: Centro de Entrenamiento}
  \LARGE

 \begin{block}{Objetivo} 
	Enfrentar exitosamente los Concursos Nacionales, Preselecci\'on Nacional 
	y \textbf{representar a Cuba en las Olimpiadas Internacionales}.
 \end{block}
 
 \end{frame}

\begin{frame}{\LARGE Las Tunas: Centro de Entrenamiento}
  \LARGE
  Profesores entrenadores:
  \begin{itemize}
    \item Orestes Landrove (Director y entrenador de Qu\'imica)
    \pause
    \item Alberto Mawad (F\'isica)
    \pause
    \item Jorge Luis Toro (Matem\'atica)
    \pause
    \item Antonio Vargas (Biolog\'ia)
    \pause
    \item Frank Arteaga (Inform\'atica)
  \end{itemize}
\end{frame}

\begin{frame}{\LARGE Las Tunas: Centro de Entrenamiento}
  \LARGE
  \begin{block}{Estructura y horario:} 
	\begin{itemize}
	 \item Dos grupos de concurso en 10mo, uno en 11no y uno en 12mo. 
	 \pause \item Dos encuentros por grado en la oncena y un encuentro com\'un.
	 \pause \item Sesiones de entrenamiento siempre por la ma\~nana.
	 \pause \item No realizan TSU. 
	 \pause \item Cuentan con el mejor claustro. 
	\end{itemize}
  \end{block}

\end{frame}

\begin{frame}{\LARGE Las Tunas: Centro de Entrenamiento}
  \Large
  \begin{block}{C\'omo se entra?} 
   \pause
	Requisitos:
	\begin{itemize}
	  \item Debes ingresar al IPVCE %Por qu'e? explicar
	  \pause
	  \item Debes vencer los ex\'amenes de habilidades
	  \pause
	  \item Debes matener 90 puntos o m\'as en en las asignaturas de permanencia
	  \pause
	  \item Debes matener m\'as de 85 puntos en el \'indice general
	  \pause
	  \item Ser disciplinado
	  %Asi Concurso -inf y Esp e Historia. la permanencia es 85%
	\end{itemize}
  \end{block}
  
\end{frame}

\begin{frame}{\LARGE Las Tunas: Centro de Entrenamiento}
  \Large
  \begin{block}{Qu\'e me aporta?} 
  Los concursos son un medio para:
   \pause
	\begin{itemize}
	  \item Probrase a s\'i mismo competitivamente
	  \pause
	  \item Sentirse parte de un colectivo con altos desempe\~nos acad\'emicos
	  \pause	  
	  \item Conocer a m\'as personas dentro y fuera del pa\'is 
	  \pause
	  \item Adentrarse en el apasionado mundo de las ciencias
	  \pause
      \item \textbf{Representar a t\'u pa\'is y decidir tu futuro profesional}
	\end{itemize}
  \end{block}
  
\end{frame}

\begin{frame}{\LARGE Las Tunas: Centro de Entrenamiento}
  \Large
  \begin{block}{Qu\'e necesito para ser exitoso en el concurso?} 
   \pause
	\begin{itemize}
	  \item Facilidad para formular problemas
	  \pause
      \item Habilidades y voluntad para resolverlos de manera ingeniosa
      \pause      	
	  \item Explotar la inteligencia colectiva (trabajo en equipo) 
	  \pause	  
	  \item \textbf{Mucho amor y dedicaci\'on por lo que haces}
	\end{itemize}
  \end{block}
  
\end{frame}


\begin{frame}{\LARGE Las Tunas: Centro de Entrenamiento}
\LARGE
\begin{block}{Y en la Universidad?} 
	
	\pause
	
	Concurso Universitario de Programaci\'on (ACM-ICPC) auspiciado por
	IBM, y en el cual Cuba participa sistem\'aticamente desde hace 6 a\~nos logrando clasificar
	equipos a la Final Mundial.
	
	\end{block}
    \begin{figure}[h]
		%\includegraphics[scale=.5]{icpclogo.png}
	\end{figure}

\end{frame}

\begin{frame}{\LARGE Las Tunas: Centro de Entrenamiento}
  \LARGE
  \begin{block}{Retos} 
   \pause
	\begin{itemize}
	  \item Mejorar el proceso de selectivo.
	  \pause
      \item Concientizar a padres y claustro de las necesidades educativas de los concursantes.
      \pause      	
	  \item Trabajar con estudiantes de Secundaria B\'asica.
	\end{itemize}
  \end{block}
  
\end{frame}

\begin{frame}{\LARGE Referencias}
\small
   \nocite{*}
   \bibliographystyle{plain}
   \bibliography{biblio}
\end{frame}



\begin{frame}{\LARGE Concursos de Conocimientos experiencia tunera}
\begin{center}
	\Huge
	Muchas Gracias
	
	\begin{figure}[h]
		\includegraphics[scale=.5]{q.png}
	\end{figure}
 \end{center} 
\end{frame}

\end{document}
